%%%%%%%%%%%%%%%%%%%%%%%%%%%%%%%%%%%%%%%%%%%%
\section{Clustering}
\label{sec:clustering}

\begin{figure}[t]
\centering
  \begin{picture}(500,170)(0,0)
    \put(0,0){\includegraphics[width=.71\columnwidth]{images/atf_wild.jpg}}
    \put(375,87){\includegraphics[width=.23\columnwidth]{images/ssc2008.jpg}}
    \put(375,0){\includegraphics[width=.23\columnwidth]{images/expe.png}}
    \put(2,153){\wl{(a)}}
    \put(377,153){\wl{(b)}}
    \put(377, 64){\wl{(c)}}    
  \end{picture}
\caption{zvtm-cluster applications running on the WILD platform (32 tiles for a total resolution of 20 480 $\times$ 6 400 pixels). (a) Zoomed-in visualization of the North-American part of the world-wide air traffic network (1~200 airports, 5~700 connections) overlaid on NASA's Blue Marble Next Generation images (86 400 $\times$ 43 200 pixels) augmented with country borders ESRI shapefiles. (b) Panning and zooming in Spitzer's Infrared Milky Way (396~032 $\times$ 12~000 pixels). (c) Controlled laboratory experiment for the evaluation of mid-air multi-scale navigation techniques \cite{nancel11}.}
\vspace{-1em}
\label{fig:atf}
\end{figure}

Besides desktop applications running on one host, ZVTM can be used to create applications whose 
rendering tasks will be split among nodes on a display cluster, such as those driving ultra-high-resolution wall-sized displays (Figure~\ref{fig:atf}).

Distributed rendering works by replicating the state of one or more VirtualSpace instances 
on the cluster. In this chapter, we introduce clustered views, which are practically the only
deviation from a ZVTM desktop application. zvtm-cluster is the graphics toolkit used in jBricks \cite{pietriga11}
for the rapid development of cluster-driven wall display user interfaces.

Dependency on \cd{zvtm-cluster} is declared as follows:

\begin{SaveVerbatim}{CodeVerb}
<dependency>
  <groupId>fr.inria.zvtm</groupId>
  <artifactId>zvtm-cluster</artifactId>
  <version>0.2.6</version> <!-- or any newer version -->
</dependency>
\end{SaveVerbatim}
\fbox{\BUseVerbatim[boxwidth=0.99\columnwidth]{CodeVerb}}

%%%%%%%%%%%%%%%%%%%%%%
\subsection{Clustered Views}

Clustered views describe the display cluster geometry, as well as a viewport within that cluster.
Views are organized around blocks (the basic unit of views). Blocks are arranged in a matrix
and numbered column-wise, starting from 0, as illustrated in Figure~\ref{fig:clview}.

\begin{figure}
\centering
 \includegraphics[width=17cm]{images/clustered_view.png}
   \caption{Clustered View Attributes}
   \label{fig:clview}
\end{figure}

%%%%%%%%%%%%%%%%%%%%%%
\subsection{Building zvtm-cluster}

\subsubsection{Basic build}

The recommended way of obtaining zvtm-cluster is from the SVN trunk. To
check out zvtm-cluster:

\codebox{svn co https://zvtm.svn.sourceforge.net/svnroot/zvtm/zvtm-cluster/trunk}

The build process is managed by \href{http://maven.apache.org/}{Maven}.
Assuming a recent version of Maven 2 or Maven 3 is installed on your machine,
executing

\codebox{mvn clean package}

at the project root will build the software and

\codebox{mvn install}

will install it in your local repository.

\subsubsection{Build options}

It is possible to build a jar containing the library, plus every
dependency. To do so, execute

\codebox{mvn assembly:assembly}

This will generate \cd{zvtm-cluster-x.y.z-jar-with-dependencies.jar}

%\subsection{Launching an executable with Maven}
%
%\codebox{mvn exec:java -Dexec.mainClass="com.example.Main" [-Dexec.args="argument1"] ... }
%
%For instance, you may launch the generic slave with
%
%\codebox{mvn exec:java -Dexec.mainClass="fr.inria.zvtm.cluster.SlaveApp"}
%
%You may also try with
%
%\begin{SaveVerbatim}{CodeVerb}
%mvn exec:exec -Dexec.executable="java"  -Dexec.args="-Xmx2g \
%    -cp %classpath fr.inria.zvtm.cluster.SlaveApp"
%\end{SaveVerbatim}
%\fbox{\BUseVerbatim[boxwidth=0.99\columnwidth]{CodeVerb}}

%%%%%%%%%%%%%%%%%%%%%%
\subsection{Features}

\subsubsection{Supported}

\begin{itemize}
\item
  Most glyphs (however VImage instances should be replaced by
  ClusteredImage instances for speed - VImage is supported but will
  usually be slower)
\item
  Multiple virtual spaces
\item
  Multiple views and cameras
\item
  Animation framework
\end{itemize}


\subsubsection{Unsupported (for now)}

\begin{itemize}
\item
  Portals (portal support is planned, for CameraPortal only at first)
\item
  Lenses (due to their implementation, supporting lenses on large
  display would be too memory-intensive)
\item
  VCursor (we are investigating VCursor support, or a suitable
  replacement; the externalization of picking now enables testing what glyphs are under the cursor {\em for a given view/slave}, but we still have to offer a mechanism that would make this transparent cluster-wide)
\end{itemize}
\subsection{Porting an application to zvtm-cluster}

Porting an existing zvtm application is done in three steps:

\begin{itemize}
\item
  If you use Maven, declare a dependency to ZVTM-cluster in your POM
  file
\end{itemize}


\begin{SaveVerbatim}{CodeVerb}
<dependency>
  <groupId>fr.inria.zvtm</groupId>
  <artifactId>zvtm-cluster</artifactId>
  <version>0.2.6</version>
</dependency>
\end{SaveVerbatim}
\fbox{\BUseVerbatim[boxwidth=0.99\columnwidth]{CodeVerb}}


\begin{itemize}
\item
  At the very beginning of your program, call \cd{VirtualSpaceManager.INSTANCE.setMaster("AppName")} where String \cd{"AppName"} matches the name passed through the \cd{-n} argument to the rendering slaves (\cd{SlaveApp}).
\item
  Create one or more \cd{ClusteredView} instances (section \ref{sec:clusterviews}).
\end{itemize}

Simple examples are located in the \cd{src/main/java/fr/inria/zvtm/cluster/examples} and are a good way to get started.

%%%%%%%%%%%%%%%%%%%%%%
\subsection{API additions to ZVTM}

\begin{itemize}
\item
  \cd{VirtualSpaceManager.setMaster()}
\item
  \cd{VirtualSpaceManager.addClusteredView()}
\item
  \cd{VirtualSpaceManager.removeClusteredView()}
\item
  \cd{VirtualSpaceManager.stop()}
\item
  \cd{ClusterGeometry} class 
\item
  \cd{ClusteredView} class 
\item
  \cd{ClusteredImage} class 
\item
  \cd{ClusteredStroke} class 
\end{itemize}

Javadoc available at \url{http://zvtm.sourceforge.net/apidocs/index.html?fr/inria/zvtm/cluster/package-summary.html}

%%%%%%%%%%%%%%%%%%%%%%
\subsection{Exclusions}

In some cases it might be necessary to declare your dependency
on zvtm-cluster in the following manner (notice the \cd{exclusion}):

\begin{SaveVerbatim}{CodeVerb}
<dependency>
   <groupId>fr.inria.zvtm</groupId>
   <artifactId>zvtm-cluster</artifactId>
   <version>0.2.7-SNAPSHOT</version>
   <exclusions>
      <exclusion>
         <groupId>fr.inria.zvtm</groupId> 
         <artifactId>zvtm-core</artifactId>
      </exclusion>
   </exclusions>
</dependency>
\end{SaveVerbatim}
\fbox{\BUseVerbatim[boxwidth=0.99\columnwidth]{CodeVerb}}

This is because zvtm-cluster depends on zvtm-core, but is not an add-on.
Rather, it replaces zvtm and augments it with clustering. Since you want
to avoid linking to the non-clustered classes in a clustered project,
you may need to declare this exclusion.

%%%%%%%%%%%%%%%%%%%%%%
\subsection{About clustered views}

\label{sec:clusterviews}

A clustered view extends the concept of a ZVTM view (i.e. a window) to
tiled display walls. A clustered view is divided into blocks. A block is
displayed by a single slave instance.

The following code shows how to create a \cd{ClusteredView} on the wall display
described in Figure \ref{fig:wildnodes}.

\begin{footnotesize}
\begin{SaveVerbatim}{CodeVerb}
vsm = VirtualSpaceManager.INSTANCE;
// the -n argument passed to SlaveApp instances must match the string passed to setMaster()
// in this example, the argument to SlaveApp would be -n WildViewer
vsm.setMaster("WildViewer");
mSpace = vsm.addVirtualSpace("fooSpace");
mCamera = mSpace.addCamera();
Vector cameras = new Vector();
cameras.add(mCamera);

// create a local view (on master computer)
mView = vsm.addFrameView(cameras, "foo", View.STD_VIEW, VIEW_W, VIEW_H, true);

// create the clustered view that will be displayed on the wall

// this wall is made of 8 x 4 30" displays (2560x1600 each,
// with approx 100px bezels -- overlay mode)
ClusterGeometry cg = new ClusterGeometry(2760, 1800, 8, 4);
Vector ccameras = new Vector();
ccameras.add(mCamera);
// the origin should be the bottom-left block of the wall
// since we have 4 rows of screens on this wall, 2nd arg = 3
ClusteredView cv = new ClusteredView(cg, 3, 8, 4, ccameras);
vsm.addClusteredView(cv);			
\end{SaveVerbatim}
\fbox{\BUseVerbatim[boxwidth=0.99\columnwidth]{CodeVerb}}
\end{footnotesize}

%%%%%%%%%%%%%%%%%%%%%%
\subsection{Displaying images with zvtm-cluster}

There are two possibilities for using images: \cd{ClusteredImage} and
the normal zvtm \cd{VImage}. \cd{ClusteredImage} is defined in the
\cd{fr.inria.zvtm.cluster} package. These images are constructed by
providing a URL. When transmitting these images to the slaves, only a
pointer (URL) is passed, and the slave is given the task of obtaining
the actual image (over an http connection, a shared drive, a common
naming scheme\ldots{}).

\verb!VImage! instances are transmitted fully to the slaves (i.e. the
bitmap is serialized). This is especially useful for resource images
(icons) or images that cannot be put on a web server / shared drive etc.
But it obviously consumes much more bandwidth.

%%%%%%%%%%%%%%%%%%%%%%
\subsection{About strokes}

It is possible to set an arbitrary \cd{Stroke} to draw the border of
a \cd{Glyph}. \cd{Stroke} being an interface, it cannot be
serialized, and several implementations, such as \cd{BasicStroke}, are
not serializable. We provide a serializable wrapper around
\cd{BasicStroke}, called \cd{ClusteredStroke}.

%%%%%%%%%%%%%%%%%%%%%%
\subsection{About paints}

It is possible to set an arbitrary \cd{Paint} to fill the inside of a
\cd{PRectangle} (and soon any \cd{ClosedShape}?). We support
\cd{GradientPaint}, \cd{RadialGradientPaint} and
\cd{LinearGradientPaint}, plus any \cd{Paint} that is serializable.

%%%%%%%%%%%%%%%%%%%%%%
\subsection{Running Applications}

zvtm-cluster applications are split in two parts:

\begin{itemize}
\item
  a master application (the one that you write)
\item
  a generic slave application \cd{fr.inria.zvtm.cluster.SlaveApp}, one
  instance of which is typically run for each screen of the display
  wall.
\end{itemize}

\textbf{All slave instances should be running before the master
application is launched.}

\begin{figure}[t]
\centering
\includegraphics[width=.9\columnwidth]{images/wild.jpg}
\caption{Wall-sized display configuration used in the following examples: an 8x4 = 32 tiles setup, driven by16 computers, each equipped with 2 graphics cards. One machine drives two 30" tiles, each tile is driven by a dedicated graphics card. The 30" tiles feature a display resolution of 2560x1600. In the following scripts, the 16 computers are named \cd{a1.foo.bar.com} to \cd{d4.foo.bar.com} and accessed with login \cd{username}.}
\vspace{-1em}
\label{fig:wildnodes}
\end{figure}

Example scripts to launch an application on an 8x4 wall, whose configuration is detailed in Figure \ref{fig:wildnodes}.

Script to launch SlaveApp on all cluster nodes:\\
\begin{tiny}
\begin{SaveVerbatim}{CodeVerb}
#!/bin/bash

function colNum {
  case "$1" in 
	  "a" ) return 0;;
	  "b" ) return 1;;
	  "c" ) return 2;;
	  "d" ) return 3;;
  esac
}

#start client nodes
for col in {a..d}
do
    for row in {1..4}
    do
        colNum $col 
        SLAVENUM1=`expr $? \* 8 + $row - 1`
        SLAVENUM2=`expr $SLAVENUM1 + 4`
        ssh username@$col$row.foo.bar.com -o UserKnownHostsFile=/dev/null -o StrictHostKeyChecking=no "sudo sysctl -w kern.ipc.maxsockbuf=80000000"
        ssh username@$col$row.foo.bar.com -o UserKnownHostsFile=/dev/null -o StrictHostKeyChecking=no "sudo sysctl -w net.inet.tcp.recvspace=40000000"
        ssh username@$col$row.foo.bar.com -o UserKnownHostsFile=/dev/null -o StrictHostKeyChecking=no "sudo sysctl -w net.inet.tcp.sendspace=40000000"
        ssh username@$col$row.foo.bar.com -o UserKnownHostsFile=/dev/null -o StrictHostKeyChecking=no \
          "cd /Users/wild/sandboxes/epietrig/zuist-cluster/trunk && java -XX:+DoEscapeAnalysis -XX:+UseConcMarkSweepGC -Xmx4g -cp target/args4j-2.0.12.jar:\
          target/aspectjrt-1.6.2.jar:target/commons-logging-1.1.jar:target/jgroups-2.7.0.GA.jar:target/log4j-1.2.14.jar:target/slf4j-api-1.5.9-RC0.jar:\
          target/slf4j-log4j12-1.5.9-RC0.jar:target/timingframework-1.0.jar:target/zvtm-cluster-0.2.7-SNAPSHOT.jar:target/zvtm-cluster-basicui-0.2.7.jar\
          fr.inria.zvtm.cluster.SlaveApp -b $SLAVENUM1 -f -d "'\\Display0' $* &

        ssh wild@$col$row.wild.lri.fr -o UserKnownHostsFile=/dev/null -o StrictHostKeyChecking=no \
          "cd /Users/wild/sandboxes/epietrig/zuist-cluster/trunk && java -XX:+DoEscapeAnalysis -XX:+UseConcMarkSweepGC -Xmx4g -cp target/args4j-2.0.12.jar:\
          target/aspectjrt-1.6.2.jar:target/commons-logging-1.1.jar:target/jgroups-2.7.0.GA.jar:target/log4j-1.2.14.jar:target/slf4j-api-1.5.9-RC0.jar:\
          target/slf4j-log4j12-1.5.9-RC0.jar:target/timingframework-1.0.jar:target/zvtm-cluster-0.2.7-SNAPSHOT.jar:target/zvtm-cluster-basicui-0.2.7.jar\
          fr.inria.zvtm.cluster.SlaveApp -b $SLAVENUM2 -f -d "'\\Display1' $* &
    done
done
\end{SaveVerbatim}
\fbox{\BUseVerbatim[boxwidth=0.99\columnwidth]{CodeVerb}}
\end{tiny}

Script to launch the main application, \cd{fr.inria.zvtm.basicui.WildViewer}, on the master node (whose IP is 192.168.0.1):\\
\begin{tiny}
\begin{SaveVerbatim}{CodeVerb}
#!/bin/bash
java -Xmx1g -Djava.net.preferIPv4Stack=true -Djgroups.bind_addr="192.168.0.1" -XX:+UseConcMarkSweepGC\
   -cp target/args4j-2.0.12.jar:target/aspectjrt-1.6.2.jar:target/commons-logging-1.1.jar:target/jgroups-2.7.0.GA.jar:target/log4j-1.2.14.jar:\
   target/slf4j-api-1.5.9-RC0.jar:target/slf4j-log4j12-1.5.9-RC0.jar:target/timingframework-1.0.jar:target/zvtm-cluster-0.2.7-SNAPSHOT.jar:\
   target/zvtm-cluster-basicui-0.2.7.jar fr.inria.zvtm.basicui.WildViewer -bw 2760 -bh 1740 -r 4 -c 8 -w 28000 -n 50 "$@"

\end{SaveVerbatim}
\fbox{\BUseVerbatim[boxwidth=0.99\columnwidth]{CodeVerb}}
\end{tiny}


%%%%%%%%%%%%%%%%%%%%%%
\subsection{Properties}

These are the system properties currently understood by zvtm-cluster.
Define them in the standard way, i.e. by using \cd{-D} on the
command line, e.g. \cd{-Dzvtm.cluster.channel\_conf="foo.xml"}

\begin{itemize}
\item
  \cd{zvtm.cluster.channel\_conf}
\end{itemize}
% \textbf{TODO: isolate properties in a package?}


%%%%%%%%%%%%%%%%%%%%%%
\subsection{Slave client options}

A summary of the options supported by SlaveApp can be displayed by
executing

\codebox{java fr.inria.zvtm.cluster.SlaveApp --help}

This summary is reproduced here:

\begin{SaveVerbatim}{CodeVerb}
Usage: SlaveApp [options] where options are:
 -b (--block) N         : view block number (slave index)
 -d (--device-name) VAL : use chosen device (fullscreen only)
 -f (--fullscreen)      : open in full screen mode
 -g (--debug)           : show ZVTM debug information
 -h (--help)            : print this help message and exit
 -n (--app-name) VAL    : application name (should match master program)
 -o                     : enable OpenGL acceleration
 -u (--undecorated)     : remove window decorations (borders)
 -x (--x-offset) N      : window x offset (ignored if fullscreen set)
 -y (--y-offset) N      : window y offset (ignored if fullscreen set)
\end{SaveVerbatim}
\fbox{\BUseVerbatim[boxwidth=0.99\columnwidth]{CodeVerb}}


%%%%%%%%%%%%%%%%%%%%%%
\subsection{OpenGL acceleration}

There are two ways of enabling OpenGL acceleration:
\begin{itemize}
\item Using Sun's OpenGL-backed Java2D rendering pipeline\footnote{\url{http://docs.oracle.com/javase/1.5.0/docs/guide/2d/new_features.html\#ogl}}, included in every JVM starting with Java 5 (available only on Windows and Linux).
\item Using Agile2D\footnote{\url{http://agile2d.sf.net}}, a Java2D rendering pipeline based on JOGL 2. This solution currently depends on JOGL 2rc5 and works on Windows, Linux and Mac OS X 10.6 or later.
\end{itemize}

The two pipelines are quite different and will provide different levels of improvement depending on the drawing primitive considered.

\subsubsection{Agile2d pipeline}

zvtm-cluster can use the Agile2d OpenGL rendering pipeline through the \cd{zvtm-agile2d} module.
To build zvtm-cluster with Agile2d support, run

\codebox{mvn -Pagile2d package}

The generic slave \cd{Agile2dSlaveApp} should be used instead of \cd{SlaveApp}.

\subsubsection{Sun OGL pipeline}

For acceleration to be enabled through the Sun Java2D OGL rendering pipeline,
you must define \cd{sun.java2d.opengl} in addition to passing the
\cd{o} option to \cd{SlaveApp}, i.e.,

\codebox{java -Dsun.java2d.opengl=True fr.inria.zvtm.cluster.SlaveApp -o (...)}

This only works under Linux and Windows (the OGL pipeline is not yet
available for Mac OS X). This is different from the OpenGL acceleration provided
by the zvtm-agile2D (JOGL-based) views. The latter works on any platform,
including Mac OS X, and does not require this property to be set to true.

\begin{enumerate}
\item The example \cd{OGLClusterExample.java} is used to trace the
OpenGL rendering capabilities with zvtm-cluster under Linux. This
example contains paths, simple polygons, text, and image samples.
With it, one can check if OpenGL calls are made for the drawing
operations associated with each of those graphical entities.

\item On each slave machine, we launch two slave applications through the
following script:

\codebox{./local\_slave\_ogl.sh -n OGLClusterExample -u -d :0.0 -b 0}

where \cd{d} indicates the screen name id and \cd{u} indicates undecorated
windows (when we cannot get fullscreen working simultaneously on two
screens).

\item In \cd{local\_slave\_ogl.sh}, when calling the java SlaveApp, we must
transmit the arguments

\codebox{-Dsun.java2d.opengl=True}

(enabling trace
of OpenGL operations) and 

\codebox{-Dsun.java2d.opengl.fbobject=false}

(desactivating the use of FrameBufferObject, otherwise the view does not properly render).

\item On the master computer, we launch the script

\codebox{./local\_master\_ogl.sh -bw 2560 -bh 1600 -r 1 -c 2}

where \cd{bw} and \cd{bh} indicate the dimensions of the window of each slave view.

\item The configuration of the slave machine in which it has been tested:
Ubuntu 9.10, NVIDIA GeForce 8800 GT/PCI/SSE2, OpenGL version 3.0.0,
driver version NVIDIA 185.18.36, GLX version 1.4.

\item Example trace messages for this configuration (we get the same output on both slaveapps):

\begin{SaveVerbatim}{CodeVerb}
sun.java2d.loops.FillRect::FillRect(AnyColor, SrcNoEa, AnyInt)
sun.java2d.loops.Blit::Blit(IntRgb, SrcNoEa, IntRgb)
sun.java2d.loops.DrawRect::DrawRect(AnyColor, SrcNoEa, AnyInt)
sun.java2d.loops.FillRect::FillRect(AnyColor, SrcNoEa, AnyInt)
sun.java2d.loops.DrawRect::DrawRect(AnyColor, SrcNoEa, AnyInt)
sun.java2d.loops.FillPath::FillPath(AnyColor, SrcNoEa, AnyInt)
sun.java2d.loops.DrawPath::DrawPath(AnyColor, SrcNoEa, AnyInt)
sun.java2d.loops.DrawGlyphList::DrawGlyphList(AnyColor, SrcNoEa, AnyInt)
sun.java2d.opengl.OGLSwToSurfaceBlit::Blit(IntRgb, AnyAlpha, "OpenGL Surface")
sun.java2d.opengl.OGLSurfaceToSurfaceBlit::Blit("OpenGL Surface", AnyAlpha,
"OpenGL Surface")
\end{SaveVerbatim}
\fbox{\BUseVerbatim[boxwidth=0.99\columnwidth]{CodeVerb}}


\item On WILD slaves (as of Feb 2011), the argument \cd{-Dsun.java2d.opengl.fbobject=true} works and the last line of the
above trace output becomes:\\
\begin{SaveVerbatim}{CodeVerb}
sun.java2d.opengl.OGLRTTSurfaceToSurfaceBlit::Blit("OpenGL Surface
(render-to-texture)", AnyAlpha, "OpenGL Surface")}
\end{SaveVerbatim}
\fbox{\BUseVerbatim[boxwidth=0.99\columnwidth]{CodeVerb}}

\end{enumerate}


%%%%%%%%%%%%%%%%%%%%%%
\subsection{Graceful shutdown}

The method \cd{VirtualSpaceManager.stop()} has been added to try to
initiate an orderly shutdown of the whole cluster.

\subsection{Logging: configuring log4j}

Since we do not want to force a particular logging library on
zvtm-cluster users, logging is done through
\href{http://www.slf4j.org/}{slf4j}. At the moment, we use the log4j
back-end to actually perform the work. A sample properties file,
\cd{log4j.properties}, is provided with zvtm-cluster. Modify it to suit
your purposes. \textbf{This properties file should be in your
classpath.}


%%%%%%%%%%%%%%%%%%%%%%
\subsection{Quirks and troubleshooting}

\begin{itemize}
\item
  When launching an application, the slave views do not display anything.
\end{itemize}
Check the application name property: the cluster name for your
application and the generic slaves should match. That is, if you called

\verb! VirtualSpaceManager.INSTANCE.setMaster("FooApp") !

in your application, you should start the generic slaves using

\verb! java fr.inria.zvtm.cluster.SlaveApp -n FooApp !

(classpath and other options omitted).

\begin{itemize}
\item
  \cd{DPath} clustering is only available when using Java 1.6+. This
  is because \cd{java.awt.geom.GeneralPath} was not serializable before Java 1.6.
\end{itemize}

%%%%%%%%%%%%%%%%%%%%%%
\subsection{Choosing the right garbage collector}

It is advisable to use the concurrent mark and sweep garbage collector,
which will minimize pause times. To do so, pass the option
\verb!-XX:+UseConcMarkSweepGC! when running both the master and slave
apps.

There is a nice
article\footnote{\url{http://www.oracle.com/technetwork/java/gc-tuning-5-138395.html}} on tuning garbage collection using the Oracle JVM.

\subsection{Generic Java troubleshooting links and tips:}

Java 6 quick troubleshooting tips\footnote{\url{http://java.sun.com/javase/6/webnotes/trouble/other/matrix6-Unix.html}}
hprof\footnote{\url{http://java.sun.com/developer/technicalArticles/Programming/HPROF.html}}

Discover hprof options by running:

\codebox{java -Xrunhprof:help}

Start VM with \verb!-XX:+HeapDumpOnOutOfMemoryError!; if
OutOfMemoryError is thrown, VM generates a heap dump.

%\subsection{Developer tips/best practices}
%
%\subsubsection{Branches}
%
%When creating a branch (for example a throwaway branch for a demo),
%rename it (change the artifact id in the maven project file). This way,
%you will avoid deploying the wrong code by mistake. Of course, if the
%branch is a feature branch that you intend to merge back into the trunk,
%change back the artifact id when merging.


%%%%%%%%%%%%%%%%%%%%%%
\subsection{Instrumenting zvtm-cluster for performance measurement}

\subsubsection{Running with YourKit Java Profiler}

YJP\footnote{\url{http://www.yourkit.com}} is a commercial Java profiler.
Instructions  needed to run programs under YJP are available\footnote{\url{http://www.yourkit.com/docs/80/help/agent.jsp}}. It mostly boils down to running

\codebox{java -agentlib:yjpagent FooClass}

after the YJP agent library directory has been added to your
\cd{LD\_LIBRARY\_PATH}. One interesting feature of YJP is that it allows
profiling of remote applications. Connecting to a remote application is
described\footnote{\url{http://www.yourkit.com/docs/80/help/connect.jsp}}.

\subsubsection{Running with VisualVM}

\href{https://visualvm.dev.java.net/}{VisualVM} allows monitoring of the
Java virtual machine that runs an application (heap size, occupancy,
thread states, \ldots{}) as well as basic CPU and memory profiling.

%To
%install VisualVM under Debian, run
%
%\verb!apt-get install visualvm !
%

Run VisualVM with

\codebox{jvisualvm}

VisualVM allows monitoring remote applications using jstatd, but not
remote profiling\footnote{\url{https://visualvm.dev.java.net/gettingstarted.html}}.

%%%%%%%%%%%%%%%%%%%%%%
\subsection{Network monitoring with JMX and JGroups}

JMX support for JGroups\footnote{\url{http://community.jboss.org/wiki/JMX}}.

