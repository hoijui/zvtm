\section{Clustering}
\label{sec:clustering}

Besides desktop applications running on one host, ZVTM can be used to create applications whose 
rendering tasks will be split among nodes on a display cluster.

Distributed rendering works by replicating the state of one or more VirtualSpace instances 
on the cluster. In this chapter, we introduce clustered views, which are practically the only
deviation from a desktop ZVTM application.

Dependency on \cd{zvtm-cluster} is declared as follows:

\begin{SaveVerbatim}{CodeVerb}
<dependency>
  <groupId>fr.inria.zvtm</groupId>
  <artifactId>zvtm-cluster</artifactId>
  <version>0.2.4-SNAPSHOT</version> <!-- or any newer version -->
</dependency>
\end{SaveVerbatim}
\fbox{\BUseVerbatim[boxwidth=0.99\columnwidth]{CodeVerb}}

\subsection{Clustered Views}

Clustered views describe the display cluster geometry, as well as a viewport within that cluster.
Views are organized around blocks (the basic unit of views). Blocks are arranged in a matrix
and numbered column-wise, starting from 0.

\begin{figure}
\centering
 \includegraphics[width=17cm]{images/clustered_view.png}
   \caption{Clustered View Attributes}
   \label{fig:clview}
\end{figure}


