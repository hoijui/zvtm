\newpage
%%%%%%%%%%%%%%%%%%%%
\section{Getting and building ZVTM}
\label{sec:build}

\subsection{Getting source code}
  
The source code is available in each zvtm-sources release on SourceForge.net\fref{http://sourceforge.net/project/showfiles.php?group_id=63244&package_id=69949} and via the SVN repository\fref{http://zvtm.sourceforge.net/download.html\#svn}.
  
\subsection{Building with Maven 2.1}
 
Maven is a very powerful management system. Visit the Maven web site to get more information\fref{http://maven.apache.org/guides/index.html}.
    
In the following we only give a brief summary of useful commands to compile and package ZVTM.

The main \cd{zvtm} directory contains a \cd{pom.xml} file that contains the Maven configuration for ZVTM. All commands are issued from that directory.

\subsubsection{Compile}

The first command in the build lifecycle compiles the source code, creating a \cd{target} directory with classes stored in \cd{target/classes}. The source code itself is located in \cd{src/main/java} with additional resources such as image icons in \cd{src/main/resources}.

\codebox{mvn compile}

\subsubsection{Package}

To build ZVTM's JAR file, issue the following command.

\codebox{mvn package}

If necessary, Maven will go through the compilation goal and all other intermediate goals before packaging. The resulting JAR file is put in directory \cd{target/}. As opposed to builds in earlier releases managed with Ant, the JAR file name now containsthe version number, such as \cd{zvtm-0.9.7.jar} or \cd{zvtm-0.9.8-SNAPSHOT.jar}.

\subsubsection{Install ZVTM in local repository}

You can install ZVTM in your local Maven repository (usually \cd{\texttildelow/.m2/repository}) with the following command:

\codebox{mvn install}

\subsubsection{Clean}

To recompile all classes, simply erase the target directory with the following command:

\codebox{mvn clean}

\subsubsection{Generating JavaDoc}

ZVTM's javadoc can be generated with the following command:

\codebox{mvn javadoc:javadoc}

The resulting \cd{apidocs/} directory is put in \cd{target/site/}.

\subsubsection{Declaring ZVTM as a dependency in your own Maven-managed project}

Declare the following repository:
\begin{SaveVerbatim}{CodeVerb}
<repository>
  <id>zvtm repository</id>
  <url>http://zvtm.sourceforge.net/maven</url>
</repository>
\end{SaveVerbatim}
\fbox{\BUseVerbatim[boxwidth=0.99\columnwidth]{CodeVerb}}

and add the following dependency in your own \cd{pom.xml} file:

\begin{SaveVerbatim}{CodeVerb}
<dependency>
  <groupId>fr.inria.zvtm</groupId>
  <artifactId>zvtm</artifactId>
  <version>0.9.8</version>
  <!-- or <version>0.10.0-SNAPSHOT</version>
       if you want to stay on the bleeding edge -->
</dependency>
\end{SaveVerbatim}
\fbox{\BUseVerbatim[boxwidth=0.99\columnwidth]{CodeVerb}}
