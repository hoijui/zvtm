\newpage
%%%%%%%%%%%%%%%%%%%%
\section{Getting and building ZVTM}
\label{sec:build}

\subsection{Getting source code}
  
The source code is available in each zvtm-sources release on SourceForge.net\fref{http://sourceforge.net/project/showfiles.php?group_id=63244&package_id=69949} and via the SVN repository\fref{http://zvtm.sourceforge.net/download.html\#svn}.

The core zvtm module is now named \cd{zvtm-core}. See section \ref{sec:modules} for a list and short description of all modules.
  
\subsection{Building with Maven 2.x or 3.x}
 
Maven is a very powerful management system. Visit the Maven web site to get more information\fref{http://maven.apache.org/guides/index.html}.
    
In the following we only give a brief summary of useful commands to compile and package ZVTM.

The root \cd{zvtm-core} directory contains a \cd{pom.xml} file that contains the Maven configuration for ZVTM. All commands are issued from that directory.

\subsubsection{Compile}

The first command in the build lifecycle compiles the source code, creating a \cd{target} directory with classes stored in \cd{target/classes}. The source code itself is located in \cd{src/main/java} with additional resources such as image icons in \cd{src/main/resources}.

\codebox{mvn compile}

\subsubsection{Package}

To build ZVTM's JAR file, issue the following command.

\codebox{mvn package}

If necessary, Maven will go through the compilation goal and all other intermediate goals before packaging. The resulting JAR file is put in directory \cd{target/}. As opposed to builds in earlier releases managed with Ant, the JAR file name now contains the version number, such as \cd{zvtm-0.10.0.jar} or \cd{zvtm-0.11.0-SNAPSHOT.jar}.

\subsubsection{Install ZVTM in local repository}

You can install ZVTM in your local Maven repository (usually \cd{\texttildelow/.m2/repository}) with the following command:

\codebox{mvn install}

\subsubsection{Clean}

To recompile all classes, simply erase the target directory with the following command:

\codebox{mvn clean}

\subsubsection{Generating JavaDoc}

\cd{zvtm-core}'s javadoc can be generated with the following command:

\codebox{mvn javadoc:javadoc}

The resulting \cd{apidocs/} directory is put in \cd{target/site/}.

When checking out the whole \cd{zvtm} project from SVN, one can generate javadoc for all modules at once (from that directory):

\codebox{mvn javadoc:aggregate}

\subsubsection{Declaring ZVTM as a dependency in your own Maven-managed project}

Declare the following repository:

\begin{SaveVerbatim}{CodeVerb}
<repository>
  <id>zvtm repository</id>
  <url>http://zvtm.sourceforge.net/maven</url>
</repository>
\end{SaveVerbatim}
\fbox{\BUseVerbatim[boxwidth=0.99\columnwidth]{CodeVerb}}

and add the following dependency in your own \cd{pom.xml} file:

\begin{SaveVerbatim}{CodeVerb}
<dependency>
  <groupId>fr.inria.zvtm</groupId>
  <artifactId>zvtm</artifactId>
  <version>0.11.0-SNAPSHOT</version>
</dependency>
\end{SaveVerbatim}
\fbox{\BUseVerbatim[boxwidth=0.99\columnwidth]{CodeVerb}}

\vspace{\parskip}

If using the SVG import/export capabilities of ZVTM, you must also declare the following dependency, as package \cd{fr.inria.zvtm.svg} was moved to a separate module in zvtm-0.10.1.

\begin{SaveVerbatim}{CodeVerb}
<dependency>
  <groupId>fr.inria.zvtm</groupId>
  <artifactId>zvtm-svg</artifactId>
  <version>0.2.0-SNAPSHOT</version>
</dependency>
\end{SaveVerbatim}
\fbox{\BUseVerbatim[boxwidth=0.99\columnwidth]{CodeVerb}}

\subsection{Available Modules}
\label{sec:modules}

The following modules provide additional features for ZVTM. All are available from SVN, and some are made available as file releases. Detailed information about each module is available from the corresponding section in this manual.

\begin{tabular}{|l|l|p{8cm}|p{2cm}|}
\hline
Module Name & Name in SVN/Maven & Description & Section \\
\hline
ZVTM & \cd{zvtm-core} & Core ZVTM module with engine, glyphs, etc. & \ref{sec:basic}, \ref{sec:animations}, \ref{sec:nav}, \ref{sec:advanced} \\
\hline
ZVTM Cluster & \cd{zvtm-cluster} & Run ZVTM applications on clusters of computers driving high-resolution large displays such as the WILD wall-sized display \url{http://insitu.lri.fr/Projects/WILD} & \ref{sec:clustering} \\
\hline
ZVTM FITS & \cd{zvtm-fits} & Display 2D FITS images in Virtual Spaces & \ref{sec:fits} \\
\hline
ZVTM Layout & \cd{zvtm-layout} & Structured data layout (trees, graphs) using Jung 2.0 & \ref{sec:jung} \\
\hline
ZVTM NodeTrix & \cd{zvtm-nodetrix} & Graph visualization with the NodeTrix visualization technique & \ref{sec:nodetrix} \\
\hline
ZVTM Treemap & \cd{zvtm-treemap} & Squarified Treemap layout (zoomable) & \ref{sec:treemap} \\
\hline
ZVTM PDF & \cd{zvtm-pdf} & Display PDF Documents in Virtual Spaces & \ref{sec:pdf} \\
\hline
ZVTM SVG & \cd{zvtm-svg} & Import/Export SVG documents to/from Virtual Spaces & \ref{sec:svg} \\
\hline
ZVTM Demos & \cd{zvtm-demos} & Basic demos of ZVTM's core features & \\
\hline
ZVTM BasicUI & \cd{zvtm-basicui} & Skeleton of a basic application that instantiates several multi-scale navigation techniques & \\
\hline
ZUIST & \cd{zuist-engine} & Multi-scale scene graph for ZVTM & \ref{sec:zuist} \\
\hline
ZUIST PDF & \cd{zuist-pdf} & Adds support for PDF documents in ZUIST scene graphs & \ref{sec:zuistpdf} \\
\hline
\end{tabular}



