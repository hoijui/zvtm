\newpage
%%%%%%%%%%%%%%%%%%%%
\section{ZUIST: a Multi-Scale Scene Manager for ZVTM}
\label{sec:zuist}

ZUIST is a generic multi-scale scene engine/API for ZVTM which makes it easier for developers to create multi-scale interfaces. Objects representing a scene at varying levels of detail are loaded/unloaded dynamically to/from memory depending on the region of the virtual space scene through the camera(s). This is similar to Google Earth, though here we are not limited to geo-spatial data but can navigate any kind of dataset, not necessarily organized as a quad-tree (see, e.g., Figure \ref{fig:ue}).

Multi-scale scenes can be built through the ZUIST API (javadoc) or can be loaded from files that describe them declaratively using a dedicated, simple XML vocabulary\footnote{\url{http://zvtm.svn.sourceforge.net/viewvc/zvtm/zuist-engine/trunk/src/main/resources/zuist_scene.dtd?view=markup}}.

For now, only the Javadoc API documentation is available, along with code examples from various applications: the basic viewer/debugger\footnote{\url{http://zvtm.svn.sourceforge.net/viewvc/zvtm/zuist-engine/trunk/src/main/java/fr/inria/zuist/viewer/}}, UIST Archive Explorer\footnote{\url{http://zvtm.svn.sourceforge.net/viewvc/zvtm/zuist-ue/trunk/}}, and Blue Marble Next Generation Explorer\footnote{\url{http://zvtm.svn.sourceforge.net/viewvc/zvtm/zuist-wm/trunk/}}.

%%%%%%%%%%
\subsection{Example Applications}

Additional information and resources about the following example applications is available on the ZUIST web page\footnote{\url{http://zvtm.sourceforge.net/zuist/}}.

%%%%%
\subsubsection{UIST Archive Explorer}

\begin{figure}[ht]
\centering
  \begin{tabular}{cc}
    \fbox{\includegraphics[height=6cm]{images/zuist1.png}}
&
    \fbox{\includegraphics[height=6cm]{images/zuist2.png}}
  \end{tabular}
 \caption{ACM UIST Archive Explorer (20th Anniversary)}
 \label{fig:ue}
\end{figure}

UIST Explorer is a ZUIST-based application that lets users navigate in 20 years of papers published at the ACM UIST conference\footnote{\url{http://www.acm.org/uist/}} from 1988 to 2007. It was one of the 20th Anniversary Interactive Visualization tools demonstrated during the 2007 conference. Users have access to 578 research papers, typically 4-to-10 page long. Papers can be browsed by year, by author, or by keywords. Overall, the multi-scale scene is composed of more than 80,000 graphical objects, most of them being 1224x1584 pixels compressed bitmap images (representing about 2GB of compressed data), loaded dynamically into memory. Running UIST Explorer typically requires less than 512MB of RAM. During the conference, the application ran on an Apple MacBook Pro laptop hooked to a SmartBoard 3000i, with a Java heap size set to 1GB, of which it never used more than 50\%.

%%%%%
\subsubsection{LRI Explorer}

\begin{figure}[ht]
\centering
  \begin{tabular}{cc}
    \fbox{\includegraphics[height=6cm]{images/zuist-lri1.png}}
&
    \fbox{\includegraphics[height=6cm]{images/zuist-lri2.png}}
\\
\\
    \fbox{\includegraphics[height=6cm]{images/zuist-lri5.png}}
&
    \fbox{\includegraphics[height=6cm]{images/zuist-lri4.png}}
  \end{tabular}
  \caption{Navigating in 1,500 documents published by LRI (Laboratoire de Recherche en Informatique, Univ. Paris-Sud)}
\end{figure}

A similar application was developed to navigate in all research documents published by members of LRI, the computer science lab at Université Paris-Sud between 2005 and 2008, ranging from book chapters, journal and conference papers to PhD dissertations and tech reports. Documents can be browsed by author or by team, and are then organized by type of publication and year of publication, down to the level where the pages of each document appear (when a PDF version is available). The total number of documents is about 1,500. A PDF version of the document is available for 753 of them, amounting to about 15,000 pages actually readable directly within the application.

%%%%%
\subsubsection{Blue Marble Next Generation Explorer}

\begin{figure}[ht]
\centering
  \begin{tabular}{c}
    \fbox{\includegraphics[height=5cm]{images/zuist3.png}}
    \fbox{\includegraphics[height=5cm]{images/zuist4.png}}
\\
\vspace{-3mm}
\\
    \fbox{\includegraphics[height=5cm]{images/zuist-atc.png}}
    \fbox{\includegraphics[height=5cm]{images/zuist-wm.png}}
  \end{tabular}
  \caption{Blue Marble Next Generation Explorer}
\end{figure}

Navigating in a multi-scale version of NASA's Blue Marble Next Generation world map\footnote{\url{http://earthobservatory.nasa.gov/Newsroom/BlueMarble/}}: 86400 x 43200 pixels decomposed into 2730 tiles (1350x1350 each) arranged in a pyramid. The map is enriched with geographical data taken from the Geonames database\footnote{\url{http://www.geonames.org/}} and boundaries for countries and administrative regions publicly available as ESRI shapefiles\footnote{\url{http://en.wikipedia.org/wiki/Shapefile}}.

%%%%%%%%%%
\subsection{Multi-scale Scenes in ZUIST}

A ZUIST scene can contain any of the following types of glyphs and their subclasses: \cd{ClosedShape}, \cd{VText}, \cd{VImage}. Support for other types of glyphs, such as \cd{DPath}, will be added in the future.

\todo{introduce concepts of levels, regions, and object descriptions. Give examples of code. Refer to javadoc.}

Dependency on \cd{zuist-engine} is declared as follows:

\begin{SaveVerbatim}{CodeVerb}
<dependency>
  <groupId>fr.inria.zvtm</groupId>
  <artifactId>zuist-engine</artifactId>
  <version>0.3.0-SNAPSHOT</version> <!-- or any newer version -->
</dependency>
\end{SaveVerbatim}
\fbox{\BUseVerbatim[boxwidth=0.99\columnwidth]{CodeVerb}}

%%%%%%%%%%
\subsection{PDF pages in ZUIST}

PDF pages in the form of ZPDFPage instances can be instantiated in a ZUIST scene. For this it is necessary to use an additional module: \cd{zuist-pdf}.

Dependency on \cd{zuist-pdf} is declared as follows:

\begin{SaveVerbatim}{CodeVerb}
<dependency>
  <groupId>fr.inria.zvtm</groupId>
  <artifactId>zuist-pdf</artifactId>
  <version>0.1.0-SNAPSHOT</version> <!-- or any newer version -->
</dependency>
\end{SaveVerbatim}
\fbox{\BUseVerbatim[boxwidth=0.99\columnwidth]{CodeVerb}}

\vspace{1em}
\label{sec:zuistpdf}

Similar to bitmap images, PDF pages are handled as ZUIST \cd{resources}. They can be created directly via the API with \cd{createResourceDescription(...)} or declared as \cd{<resource>} XML elements.

The following custom resource parameters are available:\\
\begin{tabular}{|l|l|l|}
\hline
\bf{Parameter Name} & \bf{Description} & \bf{Allowed Values} \\
\hline
\cd{pg} & page number & $[1-N]$ (where $N$ is the total number of pages\\
 &  & in the document)\\
\cd{sc} & scale in scene (multiplication factor) & $]0;+\infty[$ \\
\cd{im} & interpolation method & \cd{nearestNeighbor, bilinear, bicubic}\\
\hline
\end{tabular}
