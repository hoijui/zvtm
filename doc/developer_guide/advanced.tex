\newpage
%%%%%%%%%%%%%%%%%%%%
\section{Advanced Features}
\label{sec:advanced}

%%%%%
\subsection{Overview+Detail}
\label{sec:bifocal}

\cd{fr.inria.zvtm.engine.portals.OverviewPortal} \ldots{}

%%%%%
\subsection{Focus+Context using the Sigma Lens Framework}
\label{sec:lenses}

\begin{center}
 \includegraphics[width=16cm]{images/sigmaLensPanel2000.jpg}
\end{center}

ZVTM provides full support for Sigma Lenses, which are instantiated and programmatically manipulated in the same manner as distortion lenses described hereafter. See the Lens package API documentation for more information\fref{http://zvtm.sourceforge.net/apidocs/index.html?net/claribole/zvtm/lens/package-summary.html}. The CHI 2008 paper gives the theoretical background behind Sigma lenses. it is available from the ACM Digital Library\fref{http://doi.acm.org/10.1145/1357054.1357264} or the HAL-INRIA open archive\fref{http://hal.inria.fr/inria-00271301}. The companion video is available as a Quicktime movie\fref{http://www.lri.fr/~pietriga/2008/04/chi2008_sigma_lenses.mov}, and on YouTube\fref{http://www.youtube.com/watch?v=k3M7rty3WYs}.

A demo applet is available\fref{http://zvtm.sourceforge.net/doc/tutorials/lenses/index.html}.

Distortion lenses (also called graphical fisheye lenses) make it possible to create focus + context representations. They can be associated with any ZVTM external standard view (they cannot yet be associated with accelerated views). A lens provides a distorted representation of what is seen through all cameras of the view it is associated with.

%%%%%
\subsubsection{Lens types}

\begin{figure}
\centering
 \includegraphics[width=10cm]{images/lens_profiles.png}
   \caption{Lens magnification profiles}
   \label{fig:lprof}
\end{figure}

Lenses are defined by three properties, as shown in Figure \ref{fig:lprof}:
\begin{itemize}
\item a maximum magnification factor, which represents the magnification factor set from the center of the lens to the lens' inner radius;
\item an outer radius R1, beyond which no magnification is applied;
\item an inner radius R2, which defines the distance to the lens' center from which magnification starts to decrease from the maximum factor down to no magnification (when reaching the outer radius).
\end{itemize}

Inner and outer radii are specified in pixels. The magnification function profile, i.e., the transition from maximum magnification to no magnification at all, is defined by the lens type. Figure \ref{fig:lprof} illustrates three profiles.

The following table gives examples of available lens types. L(p) represents the distance function as defined in Carpendale and Montagnese's Framework for Unifying Presentation Space \cite{carpendale01}.


\begin{tabular}{|p{1.4cm}|c|c|c|c|}
\hline
 & $L(1)$ & $L(2)$ & $L(3)$ & $L(\inf)$ \\
\hline
Gaussian &
\includegraphics[width=3.6cm]{images/L1Gaussian.png} &
\includegraphics[width=3.6cm]{images/L2Gaussian.png} &
\includegraphics[width=3.6cm]{images/L3Gaussian.png} &
\includegraphics[width=3.6cm]{images/LInfGaussian.png} \\
\hline
Linear &
\includegraphics[width=3.6cm]{images/L1Linear.png} &
\includegraphics[width=3.6cm]{images/L2Linear.png} &
\includegraphics[width=3.6cm]{images/L3Linear.png} &
\includegraphics[width=3.6cm]{images/LInfLinear.png} \\
\hline
\vspace{-0.7cm}Inverse Cosine &
\includegraphics[width=3.6cm]{images/L1InvCos.png} &
\includegraphics[width=3.6cm]{images/L2InvCos.png} &
\includegraphics[width=3.6cm]{images/L3InvCos.png} &
\includegraphics[width=3.6cm]{images/LInfInvCos.png} \\
\hline
Manhattan &
\includegraphics[width=3.6cm]{images/L1Manhattan.png} &
\includegraphics[width=3.6cm]{images/L2Manhattan.png} &
\includegraphics[width=3.6cm]{images/L3Manhattan.png} &
\includegraphics[width=3.6cm]{images/LInfManhattan.png} \\
\hline
\end{tabular}

\subsubsection{Associating a lens with a View}

A lens is associated with a View using the View.setLens(Lens l) method. Lenses can only be associated with standard external views or applet views for now (not with accelerated views). Here is an example of code creating a fixed size lens and associating it with a view:

\begin{small}
\begin{SaveVerbatim}{CodeVerb}
View view = vsm.addExternalView(...);
// create a lens with a magnification factor of 4x, outer radius of 100px,
// inner radius of 10px (flat-top is 40px-wide)
Lens lens = view.setLens(new FSGaussianLens(4.0f, 100, 20));
\end{SaveVerbatim}
\fbox{\BUseVerbatim[boxwidth=0.99\columnwidth]{CodeVerb}}
\end{small}

\subsubsection{Unsetting a lens}

Removing a lens from a view is achieved by passing null to View.setLens(Lens l). Additionally, method dispose() should be called on the lens object to free any graphical resource used by the lens:

\begin{small}
\begin{SaveVerbatim}{CodeVerb}
view.setLens(null);
lens.dispose();
\end{SaveVerbatim}
\fbox{\BUseVerbatim[boxwidth=0.99\columnwidth]{CodeVerb}}
\end{small}

\subsubsection{Lens animation}

Each of the three lens properties (maximum magnification factor, inner and outer radii) can be changed dynamically. These changes can be animated using the Animation Manager. Here is an example of animating the previous lens' maximum magnification factor increase (from 4x to 8x in 500ms) using the linear temporal schema:

\begin{small}
\begin{SaveVerbatim}{CodeVerb}
Animation a = vsm.getAnimationManager().getAnimationFactory().createLensMagAnim(500,
           (FixedSizeLens)lens, new Float(4), true,
           IdentityInterpolator.getInstance(), null);
vsm.getAnimationManager().startAnimation(a, false);
\end{SaveVerbatim}
\fbox{\BUseVerbatim[boxwidth=0.99\columnwidth]{CodeVerb}}
\end{small}

Maximum magnification factor and radius values can be animated simultaneously. See the API Documentation for details.

\subsubsection{Using animations to achieve nice effects}

When activating a lens, a nice effect consists in creating in flat and then animating its magnification factor to the wanted value using an animation. This way, the lens seems to emerge from the previously flat surface.

\begin{small}
\begin{SaveVerbatim}{CodeVerb}
/* MAG_FACTOR is the magnification factor the lens should be set to */
float MAG_FACTOR = 4.0f;
Lens lens = new LInfFSInverseCosineLens(1.0f, 100, 50);
view.setLens(lens);
Animation a = vsm.getAnimationManager().getAnimationFactory().createLensMagAnim(500,
           (FixedSizeLens)lens, new Float(MAG_FACTOR-1), true,
           IdentityInterpolator.getInstance(), null);
vsm.getAnimationManager().startAnimation(a, false);
\end{SaveVerbatim}
\fbox{\BUseVerbatim[boxwidth=0.99\columnwidth]{CodeVerb}}
\end{small}

The opposite effect can be used when deactivatign a lens. The following animation will make it look like the lens is flattening itself. Note that in the following code we use a predefined PostAnimationAction to dispose of the lens after the animation has ended.

\begin{small}
\begin{SaveVerbatim}{CodeVerb}
void killLens(){
  lens.dispose();
  view.setLens(null);
}

...

Animation a = vsm.getAnimationManager().getAnimationFactory().createLensMagAnim(
            LENS_ANIM_TIME, (FixedSizeLens)lens,
            new Float(-MAG_FACTOR+1), true, IdentityInterpolator.getInstance(),
            new EndAction(){
                public void execute(Object subject, Animation.Dimension dimension){
                    killLens();}
                }
            );
vsm.getAnimationManager().startAnimation(a, false);
\end{SaveVerbatim}
\fbox{\BUseVerbatim[boxwidth=0.99\columnwidth]{CodeVerb}}
\end{small}

%%%%%%%%%%
\subsection{DynaSpot: Speed Dependent Area Cursor}

DynaSpot is a new pointing technique that facilitates the acquisition of small targets. It is described in detail in the following research paper:

O. Chapuis, J.-B. Labrune, E. Pietriga, DynaSpot: Speed-Dependent Area Cursor, CHI '09: Proceedings of the SIGCHI conference on Human Factors in computing systems, pages 1391-1400, April 2009, Boston, USA

The paper is available from the ACM Digital Library\footnote{\url{http://doi.acm.org/10.1145/1518701.1518911}} and from the HAL-INRIA open archive\footnote{\url{http://hal.inria.fr/inria-00373678}}.

{\bf Abstract taken from the paper:} We present DynaSpot, a new technique for acquiring targets based on the area cursor. DynaSpot couples the cursor's activation area with its speed, behaving like a point cursor at low speed or when motionless. This technique minimizes visual distraction and allows pointing anywhere in empty space without requiring an explicit mode switch, thus enabling users to perform common interactions such as region selections seamlessly. The results of our controlled experiments show that the performance of DynaSpot can be modeled by Fitts' law, and that DynaSpot significantly outperforms the point cursor and achieves, in most conditions, the same level of performance as one of the most promising techniques to date, the Bubble cursor.

In ZVTM, the DynaSpot behavior must be activated in \cd{VCursor}'s \cd{DynaPicker}, calling:

\codebox{v.getCursor().getDynaPicker().activateDynaSpot(true);}

In the \cd{mouseMoved(...)} method of the associated \cd{ViewListener} (or wherever it makes sense), call:

\codebox{vp.getVCursor().getDynaPicker().dynaPick(c); // where c is the active camera}

This updates the list of glyphs intersected by the DynaSpot circular area (disc), and identifies the one glyph actually selected (which is returned and usually highlighted in the interface).

Note: dynaPick() also gets called internally when DynaSpot's size changes.

A table on the Web site\footnote{\url{http://zvtm.sourceforge.net/doc/dynaspot.html}} summarizes the level of support for DynaSpot picking for each class of Glyph that ships with ZVTM. Subclasses will usually inherit support from the superclass, unless the actual implementation of the subclass changes dramatically from the superclass, as is the case for, e.g., \cd{VRectangle} and \cd{VRectangleOr}.

For DynaSpot to be efficient, it is necessary to give the user continuous visual feedback about the currently selected target (if any). The default behavior when DynaSpot is active in ZVTM is to highlight the glyph currently selected using its highlight color.

It is also possible to customize the behavior. This is done by implementing interface\\ \cl{fr.inria.zvtm.event.SelectionListener} and associating that listener with the ZVTM cursor's \cd{DynaPicker}:

\codebox{cursor.getDynaPicker().setSelectionListener(...)}

There is a single callback method, \cd{glyphSelected(Glyph g, boolean b)} which gets called whenever a glyph gets selected \cd{(b=true)} or unselected \cd{(b=false)} by DynaSpot. Setting your own SelectionListener completely overrides the default highlighting behavior (it does not supplement it).

%%%%%%%%%%
\subsection{Scroll Bars in Views}

\cd{fr.inria.zvtm.engine.ScrollLayer},
\cd{fr.inria.zvtm.event.ScrollListener}
\ldots{}

%%%%%
\subsection{PDF Documents}
\label{sec:pdf}

PDF documents can be rendered as ZVTM glyphs. Each page of the document corresponds to a glyph. This feature is enabled by additional module \cd{zvtm-pdf}, available from the SVN repository only at the moment. Earlier versions used Sun's pdf-renderer library\fref{https://pdf-renderer.dev.java.net}. Since v0.2.0 this module uses ICEpdf\fref{http://www.icepdf.org/} to render PDF pages. ICepdf is not available through Maven for now and has to be installed manually in your local Maven repository before you can compile \cd{zvtm-pdf}. Instructions for doing this are available in the project's \cd{pom.xml} file.

Dependency on \cd{zvtm-pdf} si declared as follows:

\begin{SaveVerbatim}{CodeVerb}
<dependency>
  <groupId>fr.inria.zvtm</groupId>
  <artifactId>zvtm-pdf</artifactId>
  <version>0.3.0</version> <!-- or any newer version -->
</dependency>
\end{SaveVerbatim}
\fbox{\BUseVerbatim[boxwidth=0.99\columnwidth]{CodeVerb}}

\vspace{1em}
PDF pages are rendered once in an offscreen \cd{BufferedImage} that then gets painted to the screen.

\subsubsection{Offscreen Rendering}

This is the preferred method. Rendering of the PDF page only happens once in the offscreen \cd{BufferedImage}. It is this image that gets painted in the rendering loop. Instantiate the PDF page as a \cd{IcePDFPageImg} glyph.

\subsubsection{Example}

The following code sample puts all pages of PDF document \cd{pdfF} in a virtual space. They are laid out horizontally with a small gap between them.

\begin{small}
\begin{SaveVerbatim}{CodeVerb}
import org.icepdf.core.pobjects.Document;
import org.icepdf.core.pobjects.Page;
import org.icepdf.core.pobjects.PDimension;
import org.icepdf.core.exceptions.PDFException;
import org.icepdf.core.exceptions.PDFSecurityException;

import fr.inria.zvtm.glyphs.IcePDFPageImg;

...

VirtualSpace vs = ...;
Document document = new Document();
try {
    document.setFile(pdfF.getAbsolutePath());
} catch (PDFException ex) {
    System.out.println("Error parsing PDF document " + ex);
} catch (PDFSecurityException ex) {
    System.out.println("Error encryption not supported " + ex);
} catch (FileNotFoundException ex) {
    System.out.println("Error file not found " + ex);
} catch (IOException ex) {
    System.out.println("Error handling PDF document " + ex);
}
int page_width = (int)document.getPageDimension(0, 0).getWidth();
for (int i = 0; i < document.getNumberOfPages(); i++) {
    IcePDFPageImg g = new IcePDFPageImg(i*page_width*1.1f*detailFactor, 0, 0,
                                        document, i, detailFactor, 1.0f);
    vs.addGlyph(g);
}
document.dispose();
\end{SaveVerbatim}
\fbox{\BUseVerbatim[boxwidth=0.99\columnwidth]{CodeVerb}}
\end{small}

\subsubsection{Advanced Functionalities}

%%%%%
\subsection{SVG Documents}
\label{sec:svg}

Dependency on \cd{zvtm-svg} is declared as follows:

\begin{SaveVerbatim}{CodeVerb}
<dependency>
  <groupId>fr.inria.zvtm</groupId>
  <artifactId>zvtm-svg</artifactId>
  <version>0.2.0</version> <!-- or any newer version -->
</dependency>
\end{SaveVerbatim}
\fbox{\BUseVerbatim[boxwidth=0.99\columnwidth]{CodeVerb}}

%%%%%
\subsection{FITS Images}
\label{sec:fits}

Dependency on \cd{zvtm-fits} is declared as follows:

\begin{SaveVerbatim}{CodeVerb}
<dependency>
  <groupId>fr.inria.zvtm</groupId>
  <artifactId>zvtm-fits</artifactId>
  <version>0.1.1-SNAPSHOT</version> <!-- or any newer version -->
</dependency>
\end{SaveVerbatim}
\fbox{\BUseVerbatim[boxwidth=0.99\columnwidth]{CodeVerb}}

%%%%%%%%%%
\subsection{Tree and Graph Layout}
\label{sec:layout}

\subsubsection{Node-link diagram layout with Jung 2.0}
\label{sec:jung}

ZVTM Layout uses the Jung 2.0\fref{http://jung.sf.net} (Java Universal Network/Graph) framework to layout complex data structures such as trees or graphs represented as ZVTM glyphs in virtual spaces.

\todo{Show screenshots, give examples}

Dependency on \cd{zvtm-layout} is declared as follows:

\begin{SaveVerbatim}{CodeVerb}
<dependency>
  <groupId>fr.inria.zvtm</groupId>
  <artifactId>zvtm-layout</artifactId>
  <version>0.3.0-SNAPSHOT</version> <!-- or any newer version -->
</dependency>
\end{SaveVerbatim}
\fbox{\BUseVerbatim[boxwidth=0.99\columnwidth]{CodeVerb}}

\subsubsection{NodeTrix}
\label{sec:nodetrix}

Implementation of the NodeTrix network visualization technique \cite{nodetrix07}, combining node-link and adjacency matrix representations.

\begin{figure}
\centering
 \includegraphics[width=\columnwidth]{images/ontotrix.png}
   \caption{Example NodeTrix-based visualization for ontologies with OntoTrix \cite{bach11}}
   \label{fig:ontotrix}
\end{figure}

Dependency on \cd{zvtm-nodetrix} is declared as follows:

\begin{SaveVerbatim}{CodeVerb}
<dependency>
  <groupId>fr.inria.zvtm</groupId>
  <artifactId>zvtm-nodetrix</artifactId>
  <version>0.4.0</version> <!-- or any newer version -->
</dependency>
\end{SaveVerbatim}
\fbox{\BUseVerbatim[boxwidth=0.99\columnwidth]{CodeVerb}}

\subsubsection{Treemaps}
\label{sec:treemap}

Implementation of the squarified treemap layout algorithm \cite{bruls99}. See Figure \ref{fig:treemap}. In addition to smooth panning and zooming (as available in any ZVTM view/camera), this implementation features some navigation features drawn from Zoomable Treemaps \cite{blanch07}, including Zoom Menus.

Dependency on \cd{zvtm-treemap} is declared as follows:

\begin{SaveVerbatim}{CodeVerb}
<dependency>
  <groupId>fr.inria.zvtm</groupId>
  <artifactId>zvtm-treemap</artifactId>
  <version>0.0.3-SNAPSHOT</version> <!-- or any newer version -->
</dependency>
\end{SaveVerbatim}
\fbox{\BUseVerbatim[boxwidth=0.99\columnwidth]{CodeVerb}}

\begin{figure}
\centering
 \includegraphics[width=.8\columnwidth]{images/treemap.jpg}
   \caption{Example Treemap used in ALMA Observatory's Operations Monitoring and Control software}
   \label{fig:treemap}
\end{figure}

The zvtm-treemap library computes the layout of a given tree within a rectangle, but does not provide a graphical representation. It is up to the user to transform the layout information into a graphical representation. Examples are provided in the package \texttt{fr.inria.zvtm.treemap.demo}.

When fitting text within a treemap rectangle, using \texttt{AdaptiveText} (part of the core set of Glyphs) may be helpful. \texttt{AdaptiveText} fits a text within a given rectangle, shortening it if necessary. Shortening rules may be specified by the user. For instance, a reasonable default would be to use as many characters as possible starting from the left, but filesystem path information might be better conveyed by keeping the rightmost information, and US states could be shortened to their two-letter abbreviation.

%%%%%
\subsection{OpenGL Java2D rendering with Agile2D}
\label{sec:agile2d}

\cd{zvtm-agile2d} provides a new type of ZVTM view that uses the OpenGL-backed Java2D renderer called Agile2D. Agile2D depends on JOGL 2.0 \footnote{\url{http://jogamp.org/jogl/www/}}.

Dependency on \cd{zvtm-agile2d} is declared as follows:

\begin{SaveVerbatim}{CodeVerb}
<dependency>
  <groupId>fr.inria.zvtm</groupId>
  <artifactId>zvtm-agile2d</artifactId>
  <version>0.0.1-SNAPSHOT</version> <!-- or any newer version -->
</dependency>
\end{SaveVerbatim}
\fbox{\BUseVerbatim[boxwidth=0.99\columnwidth]{CodeVerb}}


The client application must register the new view type at initialization in ZVTM:

\begin{small}
\begin{SaveVerbatim}{CodeVerb}
import import fr.inria.zvtm.engine.AgilePanelType;

...

View.registerViewPanelType(AgilePanelType.AGILE_VIEW, new AgilePanelType());
\end{SaveVerbatim}
\fbox{\BUseVerbatim[boxwidth=0.99\columnwidth]{CodeVerb}}
\end{small}

Then the view gets created the usual way, but using \cd{AgilePanelType.AGILE\_VIEW} as the view type:

\begin{small}
\begin{SaveVerbatim}{CodeVerb}
View v = VirtualSpaceManager.INSTANCE.addFrameView(cameras, View.ANONYMOUS,
                                              AgilePanelType.AGILE_VIEW, 800, 600, true);
\end{SaveVerbatim}
\fbox{\BUseVerbatim[boxwidth=0.99\columnwidth]{CodeVerb}}
\end{small}
