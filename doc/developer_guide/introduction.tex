\newpage
%%%%%%%%%%%%%%%%%%%%

\section*{Contributors}

\begin{itemize}
\item Emmanuel Pietriga
\item Romain Primet
\item Rodrigo A. B. de Almeida
\item Caroline Appert
\item Benjamin Bach
\item Olivier Chapuis
\item Julien Husson
\item Mathieu Nancel
\item Olivier Garaud
\item David J. Hamilton
\item Rubin Kleiman
\item Thomas Maitre
\item Eric Mounhem
\item Alex Poylisher
\item Boris Trofimov
\item Greenfreak
\item Jean-Yves Vion-Dury
\end{itemize}

\newpage
%%%%%%%%%%%%%%%%%%%%
\section{Introduction}
\label{sec:intro}

Graphical user interface toolkits such as Java Swing are very powerful, generic and portable, but are limited to conventional interface widgets of the WIMP model ({\em Windows Icons Menus and Pointers}). Implementing advanced visual interfaces for applications that work with complex, structured graphics such as graph editing applications, geographical information systems, visual language environments or information visualization tools to name a few (see Figure \ref{fig:introex} for a few examples) requires developers to use lower-level APIs such as Java2D or OpenGL. GUI toolkits tailored for the WIMP model also do not allow for the easy implementation of new interaction techniques coming from HCI ({\em Human-Computer Interaction}) research. Low-level APIs such as Java2D and OpenGL give developers much more flexibility and feature a higher expressive power, but at the cost of a much higher development and maintenance effort.

Our goal with ZVTM is to give developers a solution situated at an intermediary level of abstraction between low-level graphics APIs such as Java2D and high-level GUI toolkits such as Swing. ZVTM aims at:
\begin{itemize}
  \item making the development of interface components involving complex structured graphics easier by hiding most of the complexity (performance, memory management, UI events, geometrical transformations, animations, concurrent access);
  \item ease the robust and generic implementation of new interaction techniques coming from HCI research, and their integration in existing applications;
  \item make it possible to quickly prototype such new techniques;
  \item offer researchers a generic and efficient framework for the evaluation of these new techniques through, e.g., controlled experiments.
\end{itemize}

ZVTM provides developers with the basic components required for implementing multi-scale (or {\em zoomable}) interfaces based on a set of rich graphical instructions. ZVTM focuses on high quality visual rendering while maintaining good performance, and on the user experience by promoting foundational concepts such as {\em perceptual continuity} in graphical interfaces \cite{robertson93} through a very powerful yet simple-to-use animation module (see Section \ref{sec:animations}). ZVTM can be used to quickly implement new multi-scale navigation techniques, bi-focal representations, new pointing techniques, etc. It also features an interface with the Jung\footnote{\url{http://jung.sf.net}} graph layout framework, and facilities to easily load graphs generated by GraphViz\footnote{\url{http://www.graphviz.org}} as demonstrated in ZGRViewer\footnote{\url{http://zvtm.sf.net/zgrviewer.html}}.

ZVTM is the continuation of a project that Emmanuel Pietriga initiated with Jean-Yves Vion-Dury at Xerox Research Centre Europe while preparing his PhD. It was then called XVTM (Xerox Visual Transformation Machine). ZVTM builds upon the XVTM and is distributed under the LGPL license as was XVTM. ZVTM is now developed and maintained by INRIA project-team In Situ\footnote{\url{http://insitu.lri.fr}}, and his hosted on SourceForge.net.

Some recent examples of use of ZVTM in our research work include a new technique that facilitates pointing at small objects, called DynaSpot \cite{chapuis09}; techniques for navigating in large networks \cite{moscovich09}; the evaluation of multi-scale navigation techniques in the context of the visual search for objects with particular characteristics \cite{pietriga07}; the development of new types of magnification lenses \cite{pietriga08}. The toolkit has also been used to implement various applications, including a visual authoring tool\footnote{\url{http://www.w3.org/2001/11/IsaViz}} for Semantic Web data\cite{pietriga02,pietriga06}, visualization components for the WebContent platform\footnote{\url{http://www.webcontent.fr}}, a visual programming language for authoring XML transformations \cite{pietriga01}, a universal Gene Ontology annotation, visualization and analysis tool for functional genomics research \cite{conesa08}, Blast2GO \footnote{\url{http://bioinfo.cipf.es/blast2go/}}; a viewer for SALI networks \footnote{\url{http://sali.rguha.net/}}, a visual environment for querying RDF graphs\footnote{\url{http://rdqlplus.sourceforge.net/}}. More recently, it has been used to develop advanced multi-scale visualization components for the ALMA radio-observatory\footnote{Atacama Large Millimeter/submillimeter Array, \url{http://almaobservatory.org}} operations monitoring and control software.

% write something about Piccolo and Jazz \cite{bederson04,bederson00}

\begin{figure}[!ht]
    \begin{center}
    \begin{tabular}{ccc}
	(a) & \includegraphics[width=15cm]{images/intro_apps_top.png} & (b) \\
	(c) & \includegraphics[width=15cm]{images/intro_apps_bottom.png} & (d) \\
	(e) & \includegraphics[width=14.7cm]{images/intro_apps_alma.png} & \\
    \end{tabular}
	\caption[Various applications using ZVTM]{Various applications using ZVTM : (a) navigating in a high-resolution satellite picture (86~400 x 43~200 pixels) from NASA \cite{bluemarble} with data from \url{http://geonames.org} superimposed on it; (b) navigating in a set of 570 research papers (PDF documents) using a zoomable interface, from metadata associated with documents down to the pages of each document \cite{zuist} ; (c) Semantic Web data visual authoring tool ({\em Massachusetts Institute of Technology \& World Wide Web Consortium}) \cite{pietriga02} ; (d) universal Gene Ontology annotation, visualization and analysis tool for functional genomics research ({\em Universidad Politécnica de Valencia}) \cite{conesa08} ; (e) advanced visualization plugins for the ALMA observatory Operations Monitoring and Control software \cite{schwarz10,alma11,alma12}.}
    \label{fig:introex}
    \end{center}
\end{figure}
